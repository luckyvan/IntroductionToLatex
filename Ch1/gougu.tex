%-*- coding: UTF-8 -*-
% gougu.tex
% 勾股定理

\documentclass[UTF8]{ctexart}

\title{杂谈勾股定理}
\author{范畴}
\date{\today}

\bibliographystyle{plain}
\begin{document}
西方称勾股定理为毕达哥拉斯定理,将勾股定理的发现归功于公元前 6 世纪的毕达哥拉斯学派。该学派得到了一个法则,可以求出可排成直角三角形三边的三元数组。毕达哥拉斯学派没有书面著作,该定理的严格表述和证明则见于欧几里得《几何原本》的命题 47:“直角三角形些边上的正方形等于两直角边上的两个正方形之和。”证明是用面积做的。

我国《周髀算经》载商高(约公元前 12 世纪)答周公问
\maketitle
\tableofcontents
\section{勾股定理在古代}
\section{勾股定理的近代形式}
\bibliography{math}

\end{document}